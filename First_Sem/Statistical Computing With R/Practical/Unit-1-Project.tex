% Options for packages loaded elsewhere
\PassOptionsToPackage{unicode}{hyperref}
\PassOptionsToPackage{hyphens}{url}
%
\documentclass[
]{article}
\usepackage{amsmath,amssymb}
\usepackage{iftex}
\ifPDFTeX
  \usepackage[T1]{fontenc}
  \usepackage[utf8]{inputenc}
  \usepackage{textcomp} % provide euro and other symbols
\else % if luatex or xetex
  \usepackage{unicode-math} % this also loads fontspec
  \defaultfontfeatures{Scale=MatchLowercase}
  \defaultfontfeatures[\rmfamily]{Ligatures=TeX,Scale=1}
\fi
\usepackage{lmodern}
\ifPDFTeX\else
  % xetex/luatex font selection
\fi
% Use upquote if available, for straight quotes in verbatim environments
\IfFileExists{upquote.sty}{\usepackage{upquote}}{}
\IfFileExists{microtype.sty}{% use microtype if available
  \usepackage[]{microtype}
  \UseMicrotypeSet[protrusion]{basicmath} % disable protrusion for tt fonts
}{}
\makeatletter
\@ifundefined{KOMAClassName}{% if non-KOMA class
  \IfFileExists{parskip.sty}{%
    \usepackage{parskip}
  }{% else
    \setlength{\parindent}{0pt}
    \setlength{\parskip}{6pt plus 2pt minus 1pt}}
}{% if KOMA class
  \KOMAoptions{parskip=half}}
\makeatother
\usepackage{xcolor}
\usepackage[margin=1in]{geometry}
\usepackage{color}
\usepackage{fancyvrb}
\newcommand{\VerbBar}{|}
\newcommand{\VERB}{\Verb[commandchars=\\\{\}]}
\DefineVerbatimEnvironment{Highlighting}{Verbatim}{commandchars=\\\{\}}
% Add ',fontsize=\small' for more characters per line
\usepackage{framed}
\definecolor{shadecolor}{RGB}{248,248,248}
\newenvironment{Shaded}{\begin{snugshade}}{\end{snugshade}}
\newcommand{\AlertTok}[1]{\textcolor[rgb]{0.94,0.16,0.16}{#1}}
\newcommand{\AnnotationTok}[1]{\textcolor[rgb]{0.56,0.35,0.01}{\textbf{\textit{#1}}}}
\newcommand{\AttributeTok}[1]{\textcolor[rgb]{0.13,0.29,0.53}{#1}}
\newcommand{\BaseNTok}[1]{\textcolor[rgb]{0.00,0.00,0.81}{#1}}
\newcommand{\BuiltInTok}[1]{#1}
\newcommand{\CharTok}[1]{\textcolor[rgb]{0.31,0.60,0.02}{#1}}
\newcommand{\CommentTok}[1]{\textcolor[rgb]{0.56,0.35,0.01}{\textit{#1}}}
\newcommand{\CommentVarTok}[1]{\textcolor[rgb]{0.56,0.35,0.01}{\textbf{\textit{#1}}}}
\newcommand{\ConstantTok}[1]{\textcolor[rgb]{0.56,0.35,0.01}{#1}}
\newcommand{\ControlFlowTok}[1]{\textcolor[rgb]{0.13,0.29,0.53}{\textbf{#1}}}
\newcommand{\DataTypeTok}[1]{\textcolor[rgb]{0.13,0.29,0.53}{#1}}
\newcommand{\DecValTok}[1]{\textcolor[rgb]{0.00,0.00,0.81}{#1}}
\newcommand{\DocumentationTok}[1]{\textcolor[rgb]{0.56,0.35,0.01}{\textbf{\textit{#1}}}}
\newcommand{\ErrorTok}[1]{\textcolor[rgb]{0.64,0.00,0.00}{\textbf{#1}}}
\newcommand{\ExtensionTok}[1]{#1}
\newcommand{\FloatTok}[1]{\textcolor[rgb]{0.00,0.00,0.81}{#1}}
\newcommand{\FunctionTok}[1]{\textcolor[rgb]{0.13,0.29,0.53}{\textbf{#1}}}
\newcommand{\ImportTok}[1]{#1}
\newcommand{\InformationTok}[1]{\textcolor[rgb]{0.56,0.35,0.01}{\textbf{\textit{#1}}}}
\newcommand{\KeywordTok}[1]{\textcolor[rgb]{0.13,0.29,0.53}{\textbf{#1}}}
\newcommand{\NormalTok}[1]{#1}
\newcommand{\OperatorTok}[1]{\textcolor[rgb]{0.81,0.36,0.00}{\textbf{#1}}}
\newcommand{\OtherTok}[1]{\textcolor[rgb]{0.56,0.35,0.01}{#1}}
\newcommand{\PreprocessorTok}[1]{\textcolor[rgb]{0.56,0.35,0.01}{\textit{#1}}}
\newcommand{\RegionMarkerTok}[1]{#1}
\newcommand{\SpecialCharTok}[1]{\textcolor[rgb]{0.81,0.36,0.00}{\textbf{#1}}}
\newcommand{\SpecialStringTok}[1]{\textcolor[rgb]{0.31,0.60,0.02}{#1}}
\newcommand{\StringTok}[1]{\textcolor[rgb]{0.31,0.60,0.02}{#1}}
\newcommand{\VariableTok}[1]{\textcolor[rgb]{0.00,0.00,0.00}{#1}}
\newcommand{\VerbatimStringTok}[1]{\textcolor[rgb]{0.31,0.60,0.02}{#1}}
\newcommand{\WarningTok}[1]{\textcolor[rgb]{0.56,0.35,0.01}{\textbf{\textit{#1}}}}
\usepackage{graphicx}
\makeatletter
\def\maxwidth{\ifdim\Gin@nat@width>\linewidth\linewidth\else\Gin@nat@width\fi}
\def\maxheight{\ifdim\Gin@nat@height>\textheight\textheight\else\Gin@nat@height\fi}
\makeatother
% Scale images if necessary, so that they will not overflow the page
% margins by default, and it is still possible to overwrite the defaults
% using explicit options in \includegraphics[width, height, ...]{}
\setkeys{Gin}{width=\maxwidth,height=\maxheight,keepaspectratio}
% Set default figure placement to htbp
\makeatletter
\def\fps@figure{htbp}
\makeatother
\setlength{\emergencystretch}{3em} % prevent overfull lines
\providecommand{\tightlist}{%
  \setlength{\itemsep}{0pt}\setlength{\parskip}{0pt}}
\setcounter{secnumdepth}{-\maxdimen} % remove section numbering
\ifLuaTeX
  \usepackage{selnolig}  % disable illegal ligatures
\fi
\usepackage{bookmark}
\IfFileExists{xurl.sty}{\usepackage{xurl}}{} % add URL line breaks if available
\urlstyle{same}
\hypersetup{
  pdftitle={Unit-1 Examples and Exercises},
  pdfauthor={Santosh Kumar Sah},
  hidelinks,
  pdfcreator={LaTeX via pandoc}}

\title{Unit-1 Examples and Exercises}
\author{Santosh Kumar Sah}
\date{2025-03-11}

\begin{document}
\maketitle

\subsection{R Markdown}\label{r-markdown}

This is an R Markdown document. Markdown is a simple formatting syntax
for authoring HTML, PDF, and MS Word documents. For more details on
using R Markdown see \url{http://rmarkdown.rstudio.com}.

When you click the \textbf{Knit} button a document will be generated
that includes both content as well as the output of any embedded R code
chunks within the document. You can embed an R code chunk like this:

\begin{Shaded}
\begin{Highlighting}[]
\FunctionTok{summary}\NormalTok{(cars)}
\end{Highlighting}
\end{Shaded}

\begin{verbatim}
##      speed           dist       
##  Min.   : 4.0   Min.   :  2.00  
##  1st Qu.:12.0   1st Qu.: 26.00  
##  Median :15.0   Median : 36.00  
##  Mean   :15.4   Mean   : 42.98  
##  3rd Qu.:19.0   3rd Qu.: 56.00  
##  Max.   :25.0   Max.   :120.00
\end{verbatim}

\subsection{Including Plots}\label{including-plots}

You can also embed plots, for example:

\includegraphics{Unit-1-Project_files/figure-latex/pressure-1.pdf}

Note that the \texttt{echo\ =\ FALSE} parameter was added to the code
chunk to prevent printing of the R code that generated the plot.

\section{Session-1 Code Examples and
Exercises}\label{session-1-code-examples-and-exercises}

\subsection{Simple Data Entry in R.}\label{simple-data-entry-in-r.}

\begin{Shaded}
\begin{Highlighting}[]
\NormalTok{data }\OtherTok{\textless{}{-}} \FunctionTok{c}\NormalTok{(}\DecValTok{1}\NormalTok{,}\DecValTok{2}\NormalTok{,}\DecValTok{3}\NormalTok{,}\DecValTok{4}\NormalTok{,}\DecValTok{5}\NormalTok{,}\DecValTok{6}\NormalTok{,}\DecValTok{7}\NormalTok{,}\DecValTok{8}\NormalTok{)}
\NormalTok{data}
\end{Highlighting}
\end{Shaded}

\begin{verbatim}
## [1] 1 2 3 4 5 6 7 8
\end{verbatim}

\begin{Shaded}
\begin{Highlighting}[]
\NormalTok{text }\OtherTok{\textless{}{-}}\FunctionTok{c}\NormalTok{(}\StringTok{"a"}\NormalTok{,}\StringTok{"b"}\NormalTok{,}\StringTok{"c"}\NormalTok{,}\StringTok{\textquotesingle{}d\textquotesingle{}}\NormalTok{)}
\FunctionTok{print}\NormalTok{(text)}
\end{Highlighting}
\end{Shaded}

\begin{verbatim}
## [1] "a" "b" "c" "d"
\end{verbatim}

\begin{Shaded}
\begin{Highlighting}[]
\CommentTok{\#text2 \textless{}{-} c(a,c,v,e,f)}
\CommentTok{\#text2}

\NormalTok{data2}\OtherTok{\textless{}{-}}\FunctionTok{cbind}\NormalTok{(data,text)}

\NormalTok{data2}
\end{Highlighting}
\end{Shaded}

\begin{verbatim}
##      data text
## [1,] "1"  "a" 
## [2,] "2"  "b" 
## [3,] "3"  "c" 
## [4,] "4"  "d" 
## [5,] "5"  "a" 
## [6,] "6"  "b" 
## [7,] "7"  "c" 
## [8,] "8"  "d"
\end{verbatim}

\subsection{Array and Metrices in R.}\label{array-and-metrices-in-r.}

\begin{Shaded}
\begin{Highlighting}[]
\NormalTok{M}\OtherTok{\textless{}{-}}\FunctionTok{matrix}\NormalTok{(}
\FunctionTok{c}\NormalTok{(}\DecValTok{1}\SpecialCharTok{:}\DecValTok{9}\NormalTok{),}
\AttributeTok{nrow=}\DecValTok{3}\NormalTok{,}
\AttributeTok{ncol=}\DecValTok{3}\NormalTok{,}
\AttributeTok{byrow=}\ConstantTok{TRUE}
\NormalTok{)}

\FunctionTok{print}\NormalTok{(M)}
\end{Highlighting}
\end{Shaded}

\begin{verbatim}
##      [,1] [,2] [,3]
## [1,]    1    2    3
## [2,]    4    5    6
## [3,]    7    8    9
\end{verbatim}

\begin{Shaded}
\begin{Highlighting}[]
\NormalTok{V}\OtherTok{\textless{}{-}}\FunctionTok{c}\NormalTok{(}\DecValTok{1}\SpecialCharTok{:}\DecValTok{12}\NormalTok{)}
\NormalTok{V}
\end{Highlighting}
\end{Shaded}

\begin{verbatim}
##  [1]  1  2  3  4  5  6  7  8  9 10 11 12
\end{verbatim}

\subsubsection{MultiDimensional Array}\label{multidimensional-array}

\begin{Shaded}
\begin{Highlighting}[]
\CommentTok{\#MDA\textless{}{-}array(V,dim(2,3,))}

\CommentTok{\#MDA\textless{}{-}array(V,dim(2,3,2))}
\CommentTok{\#MDA\textless{}{-}array(V,dim=(2,3,2))}
\NormalTok{MDA}\OtherTok{\textless{}{-}}\FunctionTok{array}\NormalTok{(V, }\AttributeTok{dim =} \FunctionTok{c}\NormalTok{(}\DecValTok{2}\NormalTok{,}\DecValTok{3}\NormalTok{,}\DecValTok{2}\NormalTok{))}

\FunctionTok{print}\NormalTok{(MDA)}
\end{Highlighting}
\end{Shaded}

\begin{verbatim}
## , , 1
## 
##      [,1] [,2] [,3]
## [1,]    1    3    5
## [2,]    2    4    6
## 
## , , 2
## 
##      [,1] [,2] [,3]
## [1,]    7    9   11
## [2,]    8   10   12
\end{verbatim}

\subsection{Creating a Simple data.frame in
R.}\label{creating-a-simple-data.frame-in-r.}

\begin{Shaded}
\begin{Highlighting}[]
\NormalTok{df}\OtherTok{\textless{}{-}}\FunctionTok{data.frame}\NormalTok{(}\AttributeTok{x=}\FunctionTok{c}\NormalTok{(}\DecValTok{1}\NormalTok{,}\DecValTok{2}\NormalTok{,}\DecValTok{3}\NormalTok{),}\AttributeTok{y=}\FunctionTok{c}\NormalTok{(}\DecValTok{2}\NormalTok{,}\DecValTok{3}\NormalTok{,}\DecValTok{4}\NormalTok{),}\AttributeTok{z=}\FunctionTok{c}\NormalTok{(}\DecValTok{3}\NormalTok{,}\DecValTok{4}\NormalTok{,}\DecValTok{5}\NormalTok{))}
\NormalTok{df}
\end{Highlighting}
\end{Shaded}

\begin{verbatim}
##   x y z
## 1 1 2 3
## 2 2 3 4
## 3 3 4 5
\end{verbatim}

\begin{Shaded}
\begin{Highlighting}[]
\FunctionTok{class}\NormalTok{(df)}
\end{Highlighting}
\end{Shaded}

\begin{verbatim}
## [1] "data.frame"
\end{verbatim}

\subsubsection{A small but realistic dataframe and its
use.}\label{a-small-but-realistic-dataframe-and-its-use.}

\paragraph{create dataframe}\label{create-dataframe}

\begin{Shaded}
\begin{Highlighting}[]
\NormalTok{emp.data}\OtherTok{\textless{}{-}}\FunctionTok{data.frame}\NormalTok{(}
\AttributeTok{emp\_id=}\FunctionTok{c}\NormalTok{(}\DecValTok{1}\SpecialCharTok{:}\DecValTok{5}\NormalTok{),}
\AttributeTok{emp\_name=}\FunctionTok{c}\NormalTok{(}\StringTok{"Dyan"}\NormalTok{,}\StringTok{"Mack"}\NormalTok{,}\StringTok{"Ryan"}\NormalTok{,}\StringTok{"Gary"}\NormalTok{,}\StringTok{"Rick"}\NormalTok{),}
\AttributeTok{salary=}\FunctionTok{c}\NormalTok{(}\FloatTok{623.5}\NormalTok{,}\FloatTok{524.43}\NormalTok{,}\FloatTok{611.0}\NormalTok{,}\FloatTok{845.0}\NormalTok{,}\FloatTok{727.94}\NormalTok{),}
\AttributeTok{start\_date=}\FunctionTok{as.Date}\NormalTok{(}\FunctionTok{c}\NormalTok{(}\StringTok{"2012{-}01{-}01"}\NormalTok{,}\StringTok{"2013{-}09{-}23"}\NormalTok{,}\StringTok{"2014{-}11{-}25"}\NormalTok{,}\StringTok{"2014{-}05{-}11"}\NormalTok{,}\StringTok{"2015{-}03{-}27"}\NormalTok{)),}
\AttributeTok{stringAsFactors=}\ConstantTok{FALSE}
\NormalTok{)}


\DocumentationTok{\#\#\#\# Print The data}
\FunctionTok{print}\NormalTok{(emp.data)}
\end{Highlighting}
\end{Shaded}

\begin{verbatim}
##   emp_id emp_name salary start_date stringAsFactors
## 1      1     Dyan 623.50 2012-01-01           FALSE
## 2      2     Mack 524.43 2013-09-23           FALSE
## 3      3     Ryan 611.00 2014-11-25           FALSE
## 4      4     Gary 845.00 2014-05-11           FALSE
## 5      5     Rick 727.94 2015-03-27           FALSE
\end{verbatim}

\paragraph{Structure and Summary of the sample dataframe in
R.}\label{structure-and-summary-of-the-sample-dataframe-in-r.}

\begin{Shaded}
\begin{Highlighting}[]
\FunctionTok{print}\NormalTok{(}\FunctionTok{str}\NormalTok{(emp.data)) }\CommentTok{\# get the structure of the dataframe}
\end{Highlighting}
\end{Shaded}

\begin{verbatim}
## 'data.frame':    5 obs. of  5 variables:
##  $ emp_id         : int  1 2 3 4 5
##  $ emp_name       : chr  "Dyan" "Mack" "Ryan" "Gary" ...
##  $ salary         : num  624 524 611 845 728
##  $ start_date     : Date, format: "2012-01-01" "2013-09-23" ...
##  $ stringAsFactors: logi  FALSE FALSE FALSE FALSE FALSE
## NULL
\end{verbatim}

\begin{Shaded}
\begin{Highlighting}[]
\DocumentationTok{\#\#\#\# Print the summary of the emp.data}
\FunctionTok{print}\NormalTok{(}\FunctionTok{summary}\NormalTok{(emp.data))}
\end{Highlighting}
\end{Shaded}

\begin{verbatim}
##      emp_id    emp_name             salary        start_date        
##  Min.   :1   Length:5           Min.   :524.4   Min.   :2012-01-01  
##  1st Qu.:2   Class :character   1st Qu.:611.0   1st Qu.:2013-09-23  
##  Median :3   Mode  :character   Median :623.5   Median :2014-05-11  
##  Mean   :3                      Mean   :666.4   Mean   :2014-01-16  
##  3rd Qu.:4                      3rd Qu.:727.9   3rd Qu.:2014-11-25  
##  Max.   :5                      Max.   :845.0   Max.   :2015-03-27  
##  stringAsFactors
##  Mode :logical  
##  FALSE:5        
##                 
##                 
##                 
## 
\end{verbatim}

\paragraph{Extract part of data from dataframe in
R,}\label{extract-part-of-data-from-dataframe-in-r}

\begin{Shaded}
\begin{Highlighting}[]
\NormalTok{result}\OtherTok{\textless{}{-}}\FunctionTok{data.frame}\NormalTok{(emp.data}\SpecialCharTok{$}\NormalTok{emp\_name, emp.data}\SpecialCharTok{$}\NormalTok{salary) }\CommentTok{\# Extract specific columns}
\FunctionTok{print}\NormalTok{(result)}
\end{Highlighting}
\end{Shaded}

\begin{verbatim}
##   emp.data.emp_name emp.data.salary
## 1              Dyan          623.50
## 2              Mack          524.43
## 3              Ryan          611.00
## 4              Gary          845.00
## 5              Rick          727.94
\end{verbatim}

\begin{Shaded}
\begin{Highlighting}[]
\NormalTok{result}\OtherTok{\textless{}{-}}\NormalTok{emp.data[}\DecValTok{1}\SpecialCharTok{:}\DecValTok{2}\NormalTok{,] }\CommentTok{\# extract first two rows}
\FunctionTok{print}\NormalTok{(result)}
\end{Highlighting}
\end{Shaded}

\begin{verbatim}
##   emp_id emp_name salary start_date stringAsFactors
## 1      1     Dyan 623.50 2012-01-01           FALSE
## 2      2     Mack 524.43 2013-09-23           FALSE
\end{verbatim}

\begin{Shaded}
\begin{Highlighting}[]
\NormalTok{result}\OtherTok{\textless{}{-}}\NormalTok{emp.data[}\FunctionTok{c}\NormalTok{(}\DecValTok{3}\NormalTok{,}\DecValTok{5}\NormalTok{), }\FunctionTok{c}\NormalTok{(}\DecValTok{2}\NormalTok{,}\DecValTok{4}\NormalTok{)] }\CommentTok{\# extract 3rd and 5th row with 2nd and 4th column}
\FunctionTok{print}\NormalTok{(result)}
\end{Highlighting}
\end{Shaded}

\begin{verbatim}
##   emp_name start_date
## 3     Ryan 2014-11-25
## 5     Rick 2015-03-27
\end{verbatim}

\paragraph{Add a new column in existing
dataframe.}\label{add-a-new-column-in-existing-dataframe.}

\begin{Shaded}
\begin{Highlighting}[]
\NormalTok{emp.data}\SpecialCharTok{$}\NormalTok{dept}\OtherTok{\textless{}{-}}\FunctionTok{c}\NormalTok{(}\StringTok{"IT"}\NormalTok{,}\StringTok{"Operations"}\NormalTok{,}\StringTok{"IT"}\NormalTok{,}\StringTok{"HR"}\NormalTok{,}\StringTok{"Finance"}\NormalTok{) }\CommentTok{\# Add the \textquotesingle{}dept\textquotesingle{} column}
\NormalTok{v}\OtherTok{\textless{}{-}}\NormalTok{emp.data}
\FunctionTok{print}\NormalTok{(v)}
\end{Highlighting}
\end{Shaded}

\begin{verbatim}
##   emp_id emp_name salary start_date stringAsFactors       dept
## 1      1     Dyan 623.50 2012-01-01           FALSE         IT
## 2      2     Mack 524.43 2013-09-23           FALSE Operations
## 3      3     Ryan 611.00 2014-11-25           FALSE         IT
## 4      4     Gary 845.00 2014-05-11           FALSE         HR
## 5      5     Rick 727.94 2015-03-27           FALSE    Finance
\end{verbatim}

\paragraph{Expand dataframe in R. (Adding
Cases)}\label{expand-dataframe-in-r.-adding-cases}

\begin{Shaded}
\begin{Highlighting}[]
\NormalTok{emp.newdata}\OtherTok{\textless{}{-}}\FunctionTok{data.frame}\NormalTok{(}
\AttributeTok{emp\_id=}\FunctionTok{c}\NormalTok{(}\DecValTok{6}\SpecialCharTok{:}\DecValTok{8}\NormalTok{),}
\AttributeTok{emp\_name=}\FunctionTok{c}\NormalTok{(}\StringTok{"Rashmi"}\NormalTok{,}\StringTok{"Pranab"}\NormalTok{,}\StringTok{"Tushar"}\NormalTok{),}
\AttributeTok{salary=}\FunctionTok{c}\NormalTok{(}\FloatTok{623.5}\NormalTok{,}\FloatTok{524.43}\NormalTok{,}\FloatTok{611.0}\NormalTok{),}
\AttributeTok{start\_date=}\FunctionTok{as.Date}\NormalTok{(}\FunctionTok{c}\NormalTok{(}\StringTok{"2014{-}11{-}25"}\NormalTok{,}\StringTok{"2014{-}05{-}11"}\NormalTok{,}\StringTok{"2015{-}03{-}27"}\NormalTok{)),}
\AttributeTok{dept=}\FunctionTok{c}\NormalTok{(}\StringTok{"IT"}\NormalTok{,}\StringTok{"Operations"}\NormalTok{,}\StringTok{"Finance"}\NormalTok{),}
\AttributeTok{stringAsFactors=}\ConstantTok{FALSE}
\NormalTok{)}
\NormalTok{emp.newdata}
\end{Highlighting}
\end{Shaded}

\begin{verbatim}
##   emp_id emp_name salary start_date       dept stringAsFactors
## 1      6   Rashmi 623.50 2014-11-25         IT           FALSE
## 2      7   Pranab 524.43 2014-05-11 Operations           FALSE
## 3      8   Tushar 611.00 2015-03-27    Finance           FALSE
\end{verbatim}

\paragraph{Expand data frame in R (rbind is
used)}\label{expand-data-frame-in-r-rbind-is-used}

\begin{Shaded}
\begin{Highlighting}[]
\NormalTok{emp.finaldata}\OtherTok{\textless{}{-}}\FunctionTok{rbind}\NormalTok{(emp.data,emp.newdata)}
\NormalTok{emp.finaldata}
\end{Highlighting}
\end{Shaded}

\begin{verbatim}
##   emp_id emp_name salary start_date stringAsFactors       dept
## 1      1     Dyan 623.50 2012-01-01           FALSE         IT
## 2      2     Mack 524.43 2013-09-23           FALSE Operations
## 3      3     Ryan 611.00 2014-11-25           FALSE         IT
## 4      4     Gary 845.00 2014-05-11           FALSE         HR
## 5      5     Rick 727.94 2015-03-27           FALSE    Finance
## 6      6   Rashmi 623.50 2014-11-25           FALSE         IT
## 7      7   Pranab 524.43 2014-05-11           FALSE Operations
## 8      8   Tushar 611.00 2015-03-27           FALSE    Finance
\end{verbatim}

\section{Session-3 code examples and
exercises}\label{session-3-code-examples-and-exercises}

\subsection{Basics of R}\label{basics-of-r}

\begin{Shaded}
\begin{Highlighting}[]
\FunctionTok{print}\NormalTok{(}\DecValTok{4}\SpecialCharTok{*}\DecValTok{6}\SpecialCharTok{+}\DecValTok{5}\NormalTok{)}
\end{Highlighting}
\end{Shaded}

\begin{verbatim}
## [1] 29
\end{verbatim}

\begin{Shaded}
\begin{Highlighting}[]
\FunctionTok{print}\NormalTok{((}\DecValTok{4}\SpecialCharTok{*}\DecValTok{6}\NormalTok{)}\SpecialCharTok{+}\DecValTok{5}\NormalTok{)}
\end{Highlighting}
\end{Shaded}

\begin{verbatim}
## [1] 29
\end{verbatim}

\begin{Shaded}
\begin{Highlighting}[]
\FunctionTok{print}\NormalTok{(}\DecValTok{4}\SpecialCharTok{*}\NormalTok{(}\DecValTok{6}\SpecialCharTok{+}\DecValTok{5}\NormalTok{))}
\end{Highlighting}
\end{Shaded}

\begin{verbatim}
## [1] 44
\end{verbatim}

\begin{Shaded}
\begin{Highlighting}[]
\FunctionTok{print}\NormalTok{((}\DecValTok{4}\SpecialCharTok{+}\DecValTok{6}\NormalTok{)}\SpecialCharTok{\^{}}\DecValTok{2}\SpecialCharTok{*}\DecValTok{5}\SpecialCharTok{/}\DecValTok{10}\SpecialCharTok{+}\DecValTok{9{-}1}\NormalTok{)}
\end{Highlighting}
\end{Shaded}

\begin{verbatim}
## [1] 58
\end{verbatim}

\subsection{Variables in R. Assigning and
Removing}\label{variables-in-r.-assigning-and-removing}

\begin{Shaded}
\begin{Highlighting}[]
\NormalTok{x}\OtherTok{\textless{}{-}}\DecValTok{2}
\NormalTok{x}
\end{Highlighting}
\end{Shaded}

\begin{verbatim}
## [1] 2
\end{verbatim}

\begin{Shaded}
\begin{Highlighting}[]
\NormalTok{x}\OtherTok{=}\DecValTok{3}
\NormalTok{x}
\end{Highlighting}
\end{Shaded}

\begin{verbatim}
## [1] 3
\end{verbatim}

\begin{Shaded}
\begin{Highlighting}[]
\DecValTok{4}\OtherTok{{-}\textgreater{}}\NormalTok{x}
\NormalTok{x}
\end{Highlighting}
\end{Shaded}

\begin{verbatim}
## [1] 4
\end{verbatim}

\begin{Shaded}
\begin{Highlighting}[]
\FunctionTok{assign}\NormalTok{(}\StringTok{"x"}\NormalTok{,}\DecValTok{5}\NormalTok{)}
\NormalTok{x}
\end{Highlighting}
\end{Shaded}

\begin{verbatim}
## [1] 5
\end{verbatim}

\subsection{Data Types}\label{data-types}

\begin{Shaded}
\begin{Highlighting}[]
\NormalTok{x}\OtherTok{\textless{}{-}}\FunctionTok{c}\NormalTok{(}\DecValTok{1}\NormalTok{,}\DecValTok{2}\NormalTok{,}\DecValTok{3}\NormalTok{,}\DecValTok{4}\NormalTok{,}\DecValTok{5}\NormalTok{)}
\FunctionTok{class}\NormalTok{(x)}
\end{Highlighting}
\end{Shaded}

\begin{verbatim}
## [1] "numeric"
\end{verbatim}

\begin{Shaded}
\begin{Highlighting}[]
\NormalTok{y}\OtherTok{\textless{}{-}}\FunctionTok{c}\NormalTok{(}\DecValTok{1}\SpecialCharTok{:}\DecValTok{9}\NormalTok{)}
\FunctionTok{class}\NormalTok{(y)}
\end{Highlighting}
\end{Shaded}

\begin{verbatim}
## [1] "integer"
\end{verbatim}

\begin{Shaded}
\begin{Highlighting}[]
\NormalTok{y}\OtherTok{\textless{}{-}}\FunctionTok{c}\NormalTok{(}\DecValTok{1}\DataTypeTok{L}\SpecialCharTok{:}\DecValTok{9}\DataTypeTok{L}\NormalTok{)}
\FunctionTok{class}\NormalTok{(y)}
\end{Highlighting}
\end{Shaded}

\begin{verbatim}
## [1] "integer"
\end{verbatim}

\begin{Shaded}
\begin{Highlighting}[]
\NormalTok{y}\OtherTok{\textless{}{-}}\FunctionTok{c}\NormalTok{(}\DecValTok{1}\DataTypeTok{L}\NormalTok{,}\DecValTok{2}\DataTypeTok{L}\NormalTok{,}\DecValTok{3}\DataTypeTok{L}\NormalTok{,}\DecValTok{4}\DataTypeTok{L}\NormalTok{,}\DecValTok{5}\DataTypeTok{L}\NormalTok{)}
\FunctionTok{class}\NormalTok{(y)}
\end{Highlighting}
\end{Shaded}

\begin{verbatim}
## [1] "integer"
\end{verbatim}

\begin{Shaded}
\begin{Highlighting}[]
\FunctionTok{is.numeric}\NormalTok{(y)}
\end{Highlighting}
\end{Shaded}

\begin{verbatim}
## [1] TRUE
\end{verbatim}

\subsection{character}\label{character}

\begin{Shaded}
\begin{Highlighting}[]
\NormalTok{x}\OtherTok{\textless{}{-}}\StringTok{"data"}
\NormalTok{x}
\end{Highlighting}
\end{Shaded}

\begin{verbatim}
## [1] "data"
\end{verbatim}

\begin{Shaded}
\begin{Highlighting}[]
\FunctionTok{class}\NormalTok{(x)}
\end{Highlighting}
\end{Shaded}

\begin{verbatim}
## [1] "character"
\end{verbatim}

\begin{Shaded}
\begin{Highlighting}[]
\FunctionTok{nchar}\NormalTok{(x)}
\end{Highlighting}
\end{Shaded}

\begin{verbatim}
## [1] 4
\end{verbatim}

\subsection{Factor}\label{factor}

\begin{Shaded}
\begin{Highlighting}[]
\NormalTok{y}\OtherTok{\textless{}{-}}\FunctionTok{factor}\NormalTok{(}\StringTok{"data"}\NormalTok{)}
\NormalTok{y}
\end{Highlighting}
\end{Shaded}

\begin{verbatim}
## [1] data
## Levels: data
\end{verbatim}

\begin{Shaded}
\begin{Highlighting}[]
\FunctionTok{class}\NormalTok{(y)}
\end{Highlighting}
\end{Shaded}

\begin{verbatim}
## [1] "factor"
\end{verbatim}

\subsection{Factoris used to create and store categorical variable in
R.}\label{factoris-used-to-create-and-store-categorical-variable-in-r.}

\begin{Shaded}
\begin{Highlighting}[]
\NormalTok{gender}\OtherTok{\textless{}{-}}\FunctionTok{factor}\NormalTok{(}\FunctionTok{c}\NormalTok{(}\StringTok{"male"}\NormalTok{,}\StringTok{"female"}\NormalTok{,}\StringTok{"female"}\NormalTok{,}\StringTok{"male"}\NormalTok{))}
\FunctionTok{typeof}\NormalTok{(gender) }\CommentTok{\# datatype}
\end{Highlighting}
\end{Shaded}

\begin{verbatim}
## [1] "integer"
\end{verbatim}

\begin{Shaded}
\begin{Highlighting}[]
\FunctionTok{attributes}\NormalTok{(gender)  }\CommentTok{\# Levels and class}
\end{Highlighting}
\end{Shaded}

\begin{verbatim}
## $levels
## [1] "female" "male"  
## 
## $class
## [1] "factor"
\end{verbatim}

\begin{Shaded}
\begin{Highlighting}[]
\FunctionTok{unclass}\NormalTok{(gender) }\CommentTok{\# check how it is stored in R.}
\end{Highlighting}
\end{Shaded}

\begin{verbatim}
## [1] 2 1 1 2
## attr(,"levels")
## [1] "female" "male"
\end{verbatim}

\subsection{Date}\label{date}

\begin{Shaded}
\begin{Highlighting}[]
\NormalTok{date1}\OtherTok{\textless{}{-}}\FunctionTok{as.Date}\NormalTok{(}\StringTok{"2023{-}03{-}29"}\NormalTok{)}
\NormalTok{date1}
\end{Highlighting}
\end{Shaded}

\begin{verbatim}
## [1] "2023-03-29"
\end{verbatim}

\begin{Shaded}
\begin{Highlighting}[]
\FunctionTok{class}\NormalTok{(date1)}
\end{Highlighting}
\end{Shaded}

\begin{verbatim}
## [1] "Date"
\end{verbatim}

\begin{Shaded}
\begin{Highlighting}[]
\FunctionTok{as.numeric}\NormalTok{(date1)}
\end{Highlighting}
\end{Shaded}

\begin{verbatim}
## [1] 19445
\end{verbatim}

\begin{Shaded}
\begin{Highlighting}[]
\NormalTok{date2}\OtherTok{\textless{}{-}}\FunctionTok{as.POSIXct}\NormalTok{(}\StringTok{"2023{-}03{-}29 6:30"}\NormalTok{)}
\NormalTok{date2}
\end{Highlighting}
\end{Shaded}

\begin{verbatim}
## [1] "2023-03-29 06:30:00 +0545"
\end{verbatim}

\begin{Shaded}
\begin{Highlighting}[]
\FunctionTok{class}\NormalTok{(date2)}
\end{Highlighting}
\end{Shaded}

\begin{verbatim}
## [1] "POSIXct" "POSIXt"
\end{verbatim}

\begin{Shaded}
\begin{Highlighting}[]
\FunctionTok{as.numeric}\NormalTok{(date2)}
\end{Highlighting}
\end{Shaded}

\begin{verbatim}
## [1] 1680050700
\end{verbatim}

\subsection{Logical}\label{logical}

\begin{Shaded}
\begin{Highlighting}[]
\NormalTok{k}\OtherTok{\textless{}{-}}\ConstantTok{TRUE}
\FunctionTok{class}\NormalTok{(k)}
\end{Highlighting}
\end{Shaded}

\begin{verbatim}
## [1] "logical"
\end{verbatim}

\begin{Shaded}
\begin{Highlighting}[]
\FunctionTok{is.logical}\NormalTok{(k)}
\end{Highlighting}
\end{Shaded}

\begin{verbatim}
## [1] TRUE
\end{verbatim}

\begin{Shaded}
\begin{Highlighting}[]
\ConstantTok{TRUE}\SpecialCharTok{*}\DecValTok{5}
\end{Highlighting}
\end{Shaded}

\begin{verbatim}
## [1] 5
\end{verbatim}

\begin{Shaded}
\begin{Highlighting}[]
\DecValTok{2}\SpecialCharTok{==}\DecValTok{3}
\end{Highlighting}
\end{Shaded}

\begin{verbatim}
## [1] FALSE
\end{verbatim}

\begin{Shaded}
\begin{Highlighting}[]
\DecValTok{2}\SpecialCharTok{!=}\DecValTok{3}
\end{Highlighting}
\end{Shaded}

\begin{verbatim}
## [1] TRUE
\end{verbatim}

\begin{Shaded}
\begin{Highlighting}[]
\DecValTok{2}\SpecialCharTok{\textless{}}\DecValTok{3}
\end{Highlighting}
\end{Shaded}

\begin{verbatim}
## [1] TRUE
\end{verbatim}

\begin{Shaded}
\begin{Highlighting}[]
\StringTok{"data"}\SpecialCharTok{==}\StringTok{"stats"}
\end{Highlighting}
\end{Shaded}

\begin{verbatim}
## [1] FALSE
\end{verbatim}

\begin{Shaded}
\begin{Highlighting}[]
\StringTok{"data"}\SpecialCharTok{\textless{}}\StringTok{"stats"}
\end{Highlighting}
\end{Shaded}

\begin{verbatim}
## [1] TRUE
\end{verbatim}

\subsection{Vectors and its
operations}\label{vectors-and-its-operations}

\begin{Shaded}
\begin{Highlighting}[]
\NormalTok{x}\OtherTok{\textless{}{-}}\FunctionTok{c}\NormalTok{(}\DecValTok{1}\NormalTok{,}\DecValTok{2}\NormalTok{,}\DecValTok{3}\NormalTok{,}\DecValTok{4}\NormalTok{,}\DecValTok{5}\NormalTok{)}
\NormalTok{x}
\end{Highlighting}
\end{Shaded}

\begin{verbatim}
## [1] 1 2 3 4 5
\end{verbatim}

\begin{Shaded}
\begin{Highlighting}[]
\NormalTok{x}\SpecialCharTok{*}\DecValTok{3}
\end{Highlighting}
\end{Shaded}

\begin{verbatim}
## [1]  3  6  9 12 15
\end{verbatim}

\begin{Shaded}
\begin{Highlighting}[]
\NormalTok{x}\SpecialCharTok{+}\DecValTok{2}
\end{Highlighting}
\end{Shaded}

\begin{verbatim}
## [1] 3 4 5 6 7
\end{verbatim}

\begin{Shaded}
\begin{Highlighting}[]
\NormalTok{x}\DecValTok{{-}3}
\end{Highlighting}
\end{Shaded}

\begin{verbatim}
## [1] -2 -1  0  1  2
\end{verbatim}

\begin{Shaded}
\begin{Highlighting}[]
\NormalTok{x}\SpecialCharTok{\^{}}\DecValTok{2}
\end{Highlighting}
\end{Shaded}

\begin{verbatim}
## [1]  1  4  9 16 25
\end{verbatim}

\begin{Shaded}
\begin{Highlighting}[]
\FunctionTok{sqrt}\NormalTok{(x)}
\end{Highlighting}
\end{Shaded}

\begin{verbatim}
## [1] 1.000000 1.414214 1.732051 2.000000 2.236068
\end{verbatim}

\subsection{Two vectors of equal
length}\label{two-vectors-of-equal-length}

\begin{Shaded}
\begin{Highlighting}[]
\NormalTok{x}\OtherTok{\textless{}{-}}\DecValTok{1}\SpecialCharTok{:}\DecValTok{10}
\NormalTok{y}\OtherTok{\textless{}{-}}\SpecialCharTok{{-}}\DecValTok{5}\SpecialCharTok{:}\DecValTok{4}
\NormalTok{x}
\end{Highlighting}
\end{Shaded}

\begin{verbatim}
##  [1]  1  2  3  4  5  6  7  8  9 10
\end{verbatim}

\begin{Shaded}
\begin{Highlighting}[]
\NormalTok{y}
\end{Highlighting}
\end{Shaded}

\begin{verbatim}
##  [1] -5 -4 -3 -2 -1  0  1  2  3  4
\end{verbatim}

\begin{Shaded}
\begin{Highlighting}[]
\FunctionTok{length}\NormalTok{(x)}
\end{Highlighting}
\end{Shaded}

\begin{verbatim}
## [1] 10
\end{verbatim}

\begin{Shaded}
\begin{Highlighting}[]
\FunctionTok{length}\NormalTok{(y)}
\end{Highlighting}
\end{Shaded}

\begin{verbatim}
## [1] 10
\end{verbatim}

\begin{Shaded}
\begin{Highlighting}[]
\NormalTok{x}\SpecialCharTok{+}\NormalTok{y}
\end{Highlighting}
\end{Shaded}

\begin{verbatim}
##  [1] -4 -2  0  2  4  6  8 10 12 14
\end{verbatim}

\begin{Shaded}
\begin{Highlighting}[]
\NormalTok{x}\SpecialCharTok{{-}}\NormalTok{y}
\end{Highlighting}
\end{Shaded}

\begin{verbatim}
##  [1] 6 6 6 6 6 6 6 6 6 6
\end{verbatim}

\begin{Shaded}
\begin{Highlighting}[]
\FunctionTok{length}\NormalTok{(x}\SpecialCharTok{+}\NormalTok{y)}
\end{Highlighting}
\end{Shaded}

\begin{verbatim}
## [1] 10
\end{verbatim}

\begin{Shaded}
\begin{Highlighting}[]
\NormalTok{x}\SpecialCharTok{*}\NormalTok{y}
\end{Highlighting}
\end{Shaded}

\begin{verbatim}
##  [1] -5 -8 -9 -8 -5  0  7 16 27 40
\end{verbatim}

\begin{Shaded}
\begin{Highlighting}[]
\NormalTok{x}\SpecialCharTok{/}\NormalTok{y}
\end{Highlighting}
\end{Shaded}

\begin{verbatim}
##  [1] -0.2 -0.5 -1.0 -2.0 -5.0  Inf  7.0  4.0  3.0  2.5
\end{verbatim}

\begin{Shaded}
\begin{Highlighting}[]
\NormalTok{x}\SpecialCharTok{\^{}}\NormalTok{y}
\end{Highlighting}
\end{Shaded}

\begin{verbatim}
##  [1] 1.000000e+00 6.250000e-02 3.703704e-02 6.250000e-02 2.000000e-01
##  [6] 1.000000e+00 7.000000e+00 6.400000e+01 7.290000e+02 1.000000e+04
\end{verbatim}

\subsection{Two vectors of unequal
length}\label{two-vectors-of-unequal-length}

\begin{Shaded}
\begin{Highlighting}[]
\NormalTok{x}\OtherTok{\textless{}{-}}\DecValTok{1}\SpecialCharTok{:}\DecValTok{10}
\NormalTok{y}\OtherTok{\textless{}{-}}\FunctionTok{c}\NormalTok{(}\DecValTok{1}\NormalTok{,}\DecValTok{2}\NormalTok{)}
\NormalTok{x}\SpecialCharTok{+}\NormalTok{y}
\end{Highlighting}
\end{Shaded}

\begin{verbatim}
##  [1]  2  4  4  6  6  8  8 10 10 12
\end{verbatim}

\begin{Shaded}
\begin{Highlighting}[]
\NormalTok{z}\OtherTok{\textless{}{-}}\FunctionTok{c}\NormalTok{(}\DecValTok{1}\NormalTok{,}\DecValTok{2}\NormalTok{,}\DecValTok{3}\NormalTok{)}
\NormalTok{x}\SpecialCharTok{+}\NormalTok{z}
\end{Highlighting}
\end{Shaded}

\begin{verbatim}
## Warning in x + z: longer object length is not a multiple of shorter object
## length
\end{verbatim}

\begin{verbatim}
##  [1]  2  4  6  5  7  9  8 10 12 11
\end{verbatim}

\subsection{Comparing vectors}\label{comparing-vectors}

\begin{Shaded}
\begin{Highlighting}[]
\NormalTok{x}\OtherTok{\textless{}{-}}\DecValTok{10}\SpecialCharTok{:}\DecValTok{1}
\NormalTok{x}
\end{Highlighting}
\end{Shaded}

\begin{verbatim}
##  [1] 10  9  8  7  6  5  4  3  2  1
\end{verbatim}

\begin{Shaded}
\begin{Highlighting}[]
\NormalTok{y}\OtherTok{\textless{}{-}}\SpecialCharTok{{-}}\DecValTok{4}\SpecialCharTok{:}\DecValTok{5}
\NormalTok{y}
\end{Highlighting}
\end{Shaded}

\begin{verbatim}
##  [1] -4 -3 -2 -1  0  1  2  3  4  5
\end{verbatim}

\begin{Shaded}
\begin{Highlighting}[]
\NormalTok{x}\SpecialCharTok{\textgreater{}}\NormalTok{y}
\end{Highlighting}
\end{Shaded}

\begin{verbatim}
##  [1]  TRUE  TRUE  TRUE  TRUE  TRUE  TRUE  TRUE FALSE FALSE FALSE
\end{verbatim}

\begin{Shaded}
\begin{Highlighting}[]
\NormalTok{x}\SpecialCharTok{\textless{}=}\DecValTok{5}
\end{Highlighting}
\end{Shaded}

\begin{verbatim}
##  [1] FALSE FALSE FALSE FALSE FALSE  TRUE  TRUE  TRUE  TRUE  TRUE
\end{verbatim}

\begin{Shaded}
\begin{Highlighting}[]
\NormalTok{x}\SpecialCharTok{\textless{}}\NormalTok{y}
\end{Highlighting}
\end{Shaded}

\begin{verbatim}
##  [1] FALSE FALSE FALSE FALSE FALSE FALSE FALSE FALSE  TRUE  TRUE
\end{verbatim}

\begin{Shaded}
\begin{Highlighting}[]
\FunctionTok{any}\NormalTok{(x}\SpecialCharTok{\textless{}}\NormalTok{y)}
\end{Highlighting}
\end{Shaded}

\begin{verbatim}
## [1] TRUE
\end{verbatim}

\begin{Shaded}
\begin{Highlighting}[]
\FunctionTok{any}\NormalTok{(x}\SpecialCharTok{\textgreater{}}\NormalTok{y)}
\end{Highlighting}
\end{Shaded}

\begin{verbatim}
## [1] TRUE
\end{verbatim}

\begin{Shaded}
\begin{Highlighting}[]
\FunctionTok{all}\NormalTok{(x}\SpecialCharTok{\textgreater{}}\NormalTok{y)}
\end{Highlighting}
\end{Shaded}

\begin{verbatim}
## [1] FALSE
\end{verbatim}

\begin{Shaded}
\begin{Highlighting}[]
\FunctionTok{nchar}\NormalTok{(y)}
\end{Highlighting}
\end{Shaded}

\begin{verbatim}
##  [1] 2 2 2 2 1 1 1 1 1 1
\end{verbatim}

\begin{Shaded}
\begin{Highlighting}[]
\NormalTok{x}
\end{Highlighting}
\end{Shaded}

\begin{verbatim}
##  [1] 10  9  8  7  6  5  4  3  2  1
\end{verbatim}

\begin{Shaded}
\begin{Highlighting}[]
\NormalTok{x[}\DecValTok{1}\NormalTok{] }\CommentTok{\# retrieve first element of x}
\end{Highlighting}
\end{Shaded}

\begin{verbatim}
## [1] 10
\end{verbatim}

\begin{Shaded}
\begin{Highlighting}[]
\CommentTok{\#x[1,2] \# retrieves first and second element of x}
\NormalTok{x[}\FunctionTok{c}\NormalTok{(}\DecValTok{1}\NormalTok{,}\DecValTok{4}\NormalTok{)]}
\end{Highlighting}
\end{Shaded}

\begin{verbatim}
## [1] 10  7
\end{verbatim}

\begin{Shaded}
\begin{Highlighting}[]
\NormalTok{w}\OtherTok{\textless{}{-}}\DecValTok{1}\SpecialCharTok{:}\DecValTok{3}
\FunctionTok{names}\NormalTok{(w)}\OtherTok{\textless{}{-}}\FunctionTok{c}\NormalTok{(}\StringTok{"a"}\NormalTok{,}\StringTok{"b"}\NormalTok{,}\StringTok{"c"}\NormalTok{)}
\NormalTok{w}
\end{Highlighting}
\end{Shaded}

\begin{verbatim}
## a b c 
## 1 2 3
\end{verbatim}

\subsection{NA types missing data in
R}\label{na-types-missing-data-in-r}

\begin{Shaded}
\begin{Highlighting}[]
\NormalTok{zchar}\OtherTok{\textless{}{-}}\FunctionTok{c}\NormalTok{(}\StringTok{"Hockey"}\NormalTok{,}\ConstantTok{NA}\NormalTok{,}\StringTok{"CRicket"}\NormalTok{)}
\NormalTok{zchar}
\end{Highlighting}
\end{Shaded}

\begin{verbatim}
## [1] "Hockey"  NA        "CRicket"
\end{verbatim}

\begin{Shaded}
\begin{Highlighting}[]
\FunctionTok{nchar}\NormalTok{(z)}
\end{Highlighting}
\end{Shaded}

\begin{verbatim}
## [1] 1 1 1
\end{verbatim}

\begin{Shaded}
\begin{Highlighting}[]
\FunctionTok{is.na}\NormalTok{(zchar)}
\end{Highlighting}
\end{Shaded}

\begin{verbatim}
## [1] FALSE  TRUE FALSE
\end{verbatim}

\begin{Shaded}
\begin{Highlighting}[]
\NormalTok{z}\OtherTok{\textless{}{-}}\FunctionTok{c}\NormalTok{(}\DecValTok{1}\NormalTok{,}\DecValTok{2}\NormalTok{,}\ConstantTok{NA}\NormalTok{,}\DecValTok{4}\NormalTok{,}\DecValTok{5}\NormalTok{,}\ConstantTok{NA}\NormalTok{)}
\FunctionTok{mean}\NormalTok{(z)}
\end{Highlighting}
\end{Shaded}

\begin{verbatim}
## [1] NA
\end{verbatim}

\begin{Shaded}
\begin{Highlighting}[]
\FunctionTok{mean}\NormalTok{(z,}\AttributeTok{na.rm=}\ConstantTok{TRUE}\NormalTok{)}
\end{Highlighting}
\end{Shaded}

\begin{verbatim}
## [1] 3
\end{verbatim}

\begin{Shaded}
\begin{Highlighting}[]
\FunctionTok{var}\NormalTok{(z,}\AttributeTok{na.rm =} \ConstantTok{TRUE}\NormalTok{)}
\end{Highlighting}
\end{Shaded}

\begin{verbatim}
## [1] 3.333333
\end{verbatim}

\subsection{NULL type missing data in
R.}\label{null-type-missing-data-in-r.}

\begin{Shaded}
\begin{Highlighting}[]
\NormalTok{z}\OtherTok{\textless{}{-}}\FunctionTok{c}\NormalTok{(}\DecValTok{1}\NormalTok{,}\ConstantTok{NULL}\NormalTok{,}\DecValTok{3}\NormalTok{)}
\NormalTok{z}
\end{Highlighting}
\end{Shaded}

\begin{verbatim}
## [1] 1 3
\end{verbatim}

\begin{Shaded}
\begin{Highlighting}[]
\FunctionTok{is.null}\NormalTok{(z)}
\end{Highlighting}
\end{Shaded}

\begin{verbatim}
## [1] FALSE
\end{verbatim}

\begin{Shaded}
\begin{Highlighting}[]
\NormalTok{d}\OtherTok{\textless{}{-}}\ConstantTok{NULL}
\FunctionTok{is.null}\NormalTok{(d)}
\end{Highlighting}
\end{Shaded}

\begin{verbatim}
## [1] TRUE
\end{verbatim}

\subsection{Pipes in R}\label{pipes-in-r}

\begin{Shaded}
\begin{Highlighting}[]
\FunctionTok{library}\NormalTok{(magrittr)}
\NormalTok{x}\OtherTok{\textless{}{-}}\DecValTok{1}\SpecialCharTok{:}\DecValTok{10}
\NormalTok{x}\SpecialCharTok{\%\textgreater{}\%}\NormalTok{mean}
\end{Highlighting}
\end{Shaded}

\begin{verbatim}
## [1] 5.5
\end{verbatim}

\subsection{Chained Pipes in R}\label{chained-pipes-in-r}

\begin{Shaded}
\begin{Highlighting}[]
\NormalTok{z}\OtherTok{\textless{}{-}}\FunctionTok{c}\NormalTok{(}\DecValTok{1}\NormalTok{,}\DecValTok{2}\NormalTok{,}\ConstantTok{NA}\NormalTok{,}\DecValTok{8}\NormalTok{,}\DecValTok{3}\NormalTok{,}\ConstantTok{NA}\NormalTok{,}\DecValTok{3}\NormalTok{)}
\NormalTok{z}\SpecialCharTok{\%\textgreater{}\%}\NormalTok{is.na}\SpecialCharTok{\%\textgreater{}\%}\NormalTok{sum}
\end{Highlighting}
\end{Shaded}

\begin{verbatim}
## [1] 2
\end{verbatim}

\begin{Shaded}
\begin{Highlighting}[]
\NormalTok{z}\SpecialCharTok{\%\textgreater{}\%}\FunctionTok{mean}\NormalTok{(}\AttributeTok{na.rm =} \ConstantTok{TRUE}\NormalTok{)}
\end{Highlighting}
\end{Shaded}

\begin{verbatim}
## [1] 3.4
\end{verbatim}

\subsection{Advanced Data Structures in
R}\label{advanced-data-structures-in-r}

\begin{Shaded}
\begin{Highlighting}[]
\NormalTok{x}\OtherTok{\textless{}{-}}\DecValTok{10}\SpecialCharTok{:}\DecValTok{1}
\NormalTok{y}\SpecialCharTok{\textless{}} \SpecialCharTok{{-}}\DecValTok{4}\SpecialCharTok{:}\DecValTok{5}
\end{Highlighting}
\end{Shaded}

\begin{verbatim}
##  [1] FALSE FALSE FALSE FALSE FALSE FALSE FALSE FALSE FALSE FALSE
\end{verbatim}

\begin{Shaded}
\begin{Highlighting}[]
\NormalTok{q }\OtherTok{\textless{}{-}}\FunctionTok{c}\NormalTok{(}\StringTok{"Hockey"}\NormalTok{,}\StringTok{"Football"}\NormalTok{,}\StringTok{"Baseball"}\NormalTok{, }\StringTok{"Kabaddi"}\NormalTok{, }\StringTok{"Rugby"}\NormalTok{,}\StringTok{"Pingpong"}\NormalTok{, }\StringTok{"Basketball"}\NormalTok{,}\StringTok{"Tennis"}\NormalTok{, }\StringTok{"Cricket"}\NormalTok{, }\StringTok{"Volleyball"}\NormalTok{)}
\NormalTok{theDF }\OtherTok{\textless{}{-}}\FunctionTok{data.frame}\NormalTok{(x, y, q)}
\NormalTok{theDF}
\end{Highlighting}
\end{Shaded}

\begin{verbatim}
##     x  y          q
## 1  10 -4     Hockey
## 2   9 -3   Football
## 3   8 -2   Baseball
## 4   7 -1    Kabaddi
## 5   6  0      Rugby
## 6   5  1   Pingpong
## 7   4  2 Basketball
## 8   3  3     Tennis
## 9   2  4    Cricket
## 10  1  5 Volleyball
\end{verbatim}

\begin{Shaded}
\begin{Highlighting}[]
\NormalTok{theDF }\OtherTok{\textless{}{-}}\FunctionTok{data.frame}\NormalTok{(}\AttributeTok{First=}\NormalTok{x,}\AttributeTok{Second=}\NormalTok{y, }\AttributeTok{Sport=}\NormalTok{q)}
\FunctionTok{names}\NormalTok{(theDF)}
\end{Highlighting}
\end{Shaded}

\begin{verbatim}
## [1] "First"  "Second" "Sport"
\end{verbatim}

\begin{Shaded}
\begin{Highlighting}[]
\FunctionTok{names}\NormalTok{(theDF)[}\DecValTok{3}\NormalTok{]}
\end{Highlighting}
\end{Shaded}

\begin{verbatim}
## [1] "Sport"
\end{verbatim}

\begin{Shaded}
\begin{Highlighting}[]
\FunctionTok{rownames}\NormalTok{(theDF)}
\end{Highlighting}
\end{Shaded}

\begin{verbatim}
##  [1] "1"  "2"  "3"  "4"  "5"  "6"  "7"  "8"  "9"  "10"
\end{verbatim}

\begin{Shaded}
\begin{Highlighting}[]
\FunctionTok{rownames}\NormalTok{(theDF) }\OtherTok{\textless{}{-}} \FunctionTok{c}\NormalTok{(}\StringTok{"One"}\NormalTok{,}\StringTok{"Two"}\NormalTok{, }\StringTok{"Three"}\NormalTok{, }\StringTok{"Four"}\NormalTok{, }\StringTok{"Five"}\NormalTok{,}\StringTok{"Six"}\NormalTok{, }\StringTok{"Seven"}\NormalTok{, }\StringTok{"Eight"}\NormalTok{, }\StringTok{"Nice"}\NormalTok{,}\StringTok{"Ten"}\NormalTok{)}

\FunctionTok{rownames}\NormalTok{(theDF) }\OtherTok{\textless{}{-}} \ConstantTok{NULL}
\FunctionTok{rownames}\NormalTok{(theDF)}
\end{Highlighting}
\end{Shaded}

\begin{verbatim}
##  [1] "1"  "2"  "3"  "4"  "5"  "6"  "7"  "8"  "9"  "10"
\end{verbatim}

\begin{Shaded}
\begin{Highlighting}[]
\CommentTok{\#Printing first few rows}
\FunctionTok{head}\NormalTok{(theDF)}
\end{Highlighting}
\end{Shaded}

\begin{verbatim}
##   First Second    Sport
## 1    10     -4   Hockey
## 2     9     -3 Football
## 3     8     -2 Baseball
## 4     7     -1  Kabaddi
## 5     6      0    Rugby
## 6     5      1 Pingpong
\end{verbatim}

\begin{Shaded}
\begin{Highlighting}[]
\CommentTok{\#Printing first seven rows}
\FunctionTok{head}\NormalTok{(theDF, }\AttributeTok{n=}\DecValTok{7}\NormalTok{)}
\end{Highlighting}
\end{Shaded}

\begin{verbatim}
##   First Second      Sport
## 1    10     -4     Hockey
## 2     9     -3   Football
## 3     8     -2   Baseball
## 4     7     -1    Kabaddi
## 5     6      0      Rugby
## 6     5      1   Pingpong
## 7     4      2 Basketball
\end{verbatim}

\begin{Shaded}
\begin{Highlighting}[]
\CommentTok{\#Printing last few rows}
\FunctionTok{tail}\NormalTok{(theDF)}
\end{Highlighting}
\end{Shaded}

\begin{verbatim}
##    First Second      Sport
## 5      6      0      Rugby
## 6      5      1   Pingpong
## 7      4      2 Basketball
## 8      3      3     Tennis
## 9      2      4    Cricket
## 10     1      5 Volleyball
\end{verbatim}

\begin{Shaded}
\begin{Highlighting}[]
\FunctionTok{class}\NormalTok{(theDF)}
\end{Highlighting}
\end{Shaded}

\begin{verbatim}
## [1] "data.frame"
\end{verbatim}

\begin{Shaded}
\begin{Highlighting}[]
\CommentTok{\#Structure of data frame by}
\CommentTok{\#variables}
\FunctionTok{str}\NormalTok{(theDF)}
\end{Highlighting}
\end{Shaded}

\begin{verbatim}
## 'data.frame':    10 obs. of  3 variables:
##  $ First : int  10 9 8 7 6 5 4 3 2 1
##  $ Second: int  -4 -3 -2 -1 0 1 2 3 4 5
##  $ Sport : chr  "Hockey" "Football" "Baseball" "Kabaddi" ...
\end{verbatim}

\begin{Shaded}
\begin{Highlighting}[]
\NormalTok{theDF[}\DecValTok{3}\NormalTok{,}\DecValTok{2}\NormalTok{]; theDF[}\DecValTok{3}\NormalTok{, }\DecValTok{2}\SpecialCharTok{:}\DecValTok{3}\NormalTok{]}
\end{Highlighting}
\end{Shaded}

\begin{verbatim}
## [1] -2
\end{verbatim}

\begin{verbatim}
##   Second    Sport
## 3     -2 Baseball
\end{verbatim}

\begin{Shaded}
\begin{Highlighting}[]
\NormalTok{theDF[, }\DecValTok{3}\NormalTok{]; theDF[}\DecValTok{3}\NormalTok{,]}
\end{Highlighting}
\end{Shaded}

\begin{verbatim}
##  [1] "Hockey"     "Football"   "Baseball"   "Kabaddi"    "Rugby"     
##  [6] "Pingpong"   "Basketball" "Tennis"     "Cricket"    "Volleyball"
\end{verbatim}

\begin{verbatim}
##   First Second    Sport
## 3     8     -2 Baseball
\end{verbatim}

\begin{Shaded}
\begin{Highlighting}[]
\NormalTok{theDF[, }\FunctionTok{c}\NormalTok{(}\StringTok{"First"}\NormalTok{, }\StringTok{"Sport"}\NormalTok{)]}
\end{Highlighting}
\end{Shaded}

\begin{verbatim}
##    First      Sport
## 1     10     Hockey
## 2      9   Football
## 3      8   Baseball
## 4      7    Kabaddi
## 5      6      Rugby
## 6      5   Pingpong
## 7      4 Basketball
## 8      3     Tennis
## 9      2    Cricket
## 10     1 Volleyball
\end{verbatim}

\begin{Shaded}
\begin{Highlighting}[]
\NormalTok{theDF[, }\StringTok{"Sport"}\NormalTok{, drop}\OtherTok{=}\ConstantTok{FALSE}\NormalTok{]}
\end{Highlighting}
\end{Shaded}

\begin{verbatim}
##         Sport
## 1      Hockey
## 2    Football
## 3    Baseball
## 4     Kabaddi
## 5       Rugby
## 6    Pingpong
## 7  Basketball
## 8      Tennis
## 9     Cricket
## 10 Volleyball
\end{verbatim}

\subsection{Lists in R}\label{lists-in-r}

\begin{Shaded}
\begin{Highlighting}[]
\CommentTok{\#Three element list}
\NormalTok{list1 }\OtherTok{\textless{}{-}} \FunctionTok{list}\NormalTok{(}\DecValTok{1}\NormalTok{,}\DecValTok{2}\NormalTok{,}\DecValTok{3}\NormalTok{)}
\CommentTok{\#Single element list}
\NormalTok{list2 }\OtherTok{\textless{}{-}} \FunctionTok{list}\NormalTok{(}\FunctionTok{c}\NormalTok{(}\DecValTok{1}\NormalTok{,}\DecValTok{2}\NormalTok{,}\DecValTok{3}\NormalTok{))}
\CommentTok{\#Two vector list}
\NormalTok{list3 }\OtherTok{\textless{}{-}} \FunctionTok{list}\NormalTok{(}\FunctionTok{c}\NormalTok{(}\DecValTok{1}\NormalTok{,}\DecValTok{2}\NormalTok{,}\DecValTok{3}\NormalTok{), }\DecValTok{3}\SpecialCharTok{:}\DecValTok{7}\NormalTok{)}
\CommentTok{\#List with data.frame and vector}
\NormalTok{list4 }\OtherTok{\textless{}{-}} \FunctionTok{list}\NormalTok{(theDF, }\DecValTok{1}\SpecialCharTok{:}\DecValTok{10}\NormalTok{)}
\CommentTok{\#Three element list}
\NormalTok{list5 }\OtherTok{\textless{}{-}} \FunctionTok{list}\NormalTok{(theDF, }\DecValTok{1}\SpecialCharTok{:}\DecValTok{10}\NormalTok{, list3)}
\CommentTok{\#Names of the list}
\FunctionTok{names}\NormalTok{(list5)}
\end{Highlighting}
\end{Shaded}

\begin{verbatim}
## NULL
\end{verbatim}

\begin{Shaded}
\begin{Highlighting}[]
\FunctionTok{names}\NormalTok{(list5) }\OtherTok{\textless{}{-}}\FunctionTok{c}\NormalTok{(}\StringTok{"data.frame"}\NormalTok{,}\StringTok{"vector"}\NormalTok{, }\StringTok{"list"}\NormalTok{)}
\FunctionTok{names}\NormalTok{(list5)}
\end{Highlighting}
\end{Shaded}

\begin{verbatim}
## [1] "data.frame" "vector"     "list"
\end{verbatim}

\begin{Shaded}
\begin{Highlighting}[]
\NormalTok{list5}
\end{Highlighting}
\end{Shaded}

\begin{verbatim}
## $data.frame
##    First Second      Sport
## 1     10     -4     Hockey
## 2      9     -3   Football
## 3      8     -2   Baseball
## 4      7     -1    Kabaddi
## 5      6      0      Rugby
## 6      5      1   Pingpong
## 7      4      2 Basketball
## 8      3      3     Tennis
## 9      2      4    Cricket
## 10     1      5 Volleyball
## 
## $vector
##  [1]  1  2  3  4  5  6  7  8  9 10
## 
## $list
## $list[[1]]
## [1] 1 2 3
## 
## $list[[2]]
## [1] 3 4 5 6 7
\end{verbatim}

\begin{Shaded}
\begin{Highlighting}[]
\NormalTok{list6 }\OtherTok{\textless{}{-}} \FunctionTok{list}\NormalTok{(}\AttributeTok{TheDataFrame=}\NormalTok{theDF,}
\AttributeTok{TheVector=}\DecValTok{1}\SpecialCharTok{:}\DecValTok{10}\NormalTok{, }\AttributeTok{TheList=}\NormalTok{list3)}
\FunctionTok{names}\NormalTok{(list6)}
\end{Highlighting}
\end{Shaded}

\begin{verbatim}
## [1] "TheDataFrame" "TheVector"    "TheList"
\end{verbatim}

\subsection{Access Elements of Lists}\label{access-elements-of-lists}

\begin{Shaded}
\begin{Highlighting}[]
\CommentTok{\#Use double square brackets}

\CommentTok{\#Specify either the element number or name}
\NormalTok{list5[[}\DecValTok{1}\NormalTok{]]}
\end{Highlighting}
\end{Shaded}

\begin{verbatim}
##    First Second      Sport
## 1     10     -4     Hockey
## 2      9     -3   Football
## 3      8     -2   Baseball
## 4      7     -1    Kabaddi
## 5      6      0      Rugby
## 6      5      1   Pingpong
## 7      4      2 Basketball
## 8      3      3     Tennis
## 9      2      4    Cricket
## 10     1      5 Volleyball
\end{verbatim}

\begin{Shaded}
\begin{Highlighting}[]
\NormalTok{list5[[}\StringTok{"data.frame"}\NormalTok{]]}
\end{Highlighting}
\end{Shaded}

\begin{verbatim}
##    First Second      Sport
## 1     10     -4     Hockey
## 2      9     -3   Football
## 3      8     -2   Baseball
## 4      7     -1    Kabaddi
## 5      6      0      Rugby
## 6      5      1   Pingpong
## 7      4      2 Basketball
## 8      3      3     Tennis
## 9      2      4    Cricket
## 10     1      5 Volleyball
\end{verbatim}

\begin{Shaded}
\begin{Highlighting}[]
\CommentTok{\# This allows access to only one element at a time}

\CommentTok{\#Accessed element manipulation}
\NormalTok{list5[[}\DecValTok{1}\NormalTok{]]}\SpecialCharTok{$}\NormalTok{Sport }\CommentTok{\#Sport variable}
\end{Highlighting}
\end{Shaded}

\begin{verbatim}
##  [1] "Hockey"     "Football"   "Baseball"   "Kabaddi"    "Rugby"     
##  [6] "Pingpong"   "Basketball" "Tennis"     "Cricket"    "Volleyball"
\end{verbatim}

\begin{Shaded}
\begin{Highlighting}[]
\NormalTok{list5[[}\DecValTok{1}\NormalTok{]][, }\StringTok{"Second"}\NormalTok{]}
\end{Highlighting}
\end{Shaded}

\begin{verbatim}
##  [1] -4 -3 -2 -1  0  1  2  3  4  5
\end{verbatim}

\begin{Shaded}
\begin{Highlighting}[]
\NormalTok{list5[[}\DecValTok{1}\NormalTok{]][, }\StringTok{"Second"}\NormalTok{, drop}\OtherTok{=}\NormalTok{F]}
\end{Highlighting}
\end{Shaded}

\begin{verbatim}
##    Second
## 1      -4
## 2      -3
## 3      -2
## 4      -1
## 5       0
## 6       1
## 7       2
## 8       3
## 9       4
## 10      5
\end{verbatim}

\begin{Shaded}
\begin{Highlighting}[]
\FunctionTok{length}\NormalTok{(list5)}
\end{Highlighting}
\end{Shaded}

\begin{verbatim}
## [1] 3
\end{verbatim}

\begin{Shaded}
\begin{Highlighting}[]
\CommentTok{\#Adding new element}
\NormalTok{list5[[}\DecValTok{4}\NormalTok{]] }\OtherTok{\textless{}{-}} \DecValTok{2}
\NormalTok{list5[[}\StringTok{"NewElement"}\NormalTok{]] }\OtherTok{\textless{}{-}}\DecValTok{3}\SpecialCharTok{:}\DecValTok{6}
\end{Highlighting}
\end{Shaded}

\subsection{Matrices in R}\label{matrices-in-r}

\begin{Shaded}
\begin{Highlighting}[]
\NormalTok{A }\OtherTok{\textless{}{-}} \FunctionTok{matrix}\NormalTok{(}\DecValTok{1}\SpecialCharTok{:}\DecValTok{10}\NormalTok{, }\AttributeTok{nrow=}\DecValTok{5}\NormalTok{)}
\NormalTok{B }\OtherTok{\textless{}{-}} \FunctionTok{matrix}\NormalTok{(}\DecValTok{21}\SpecialCharTok{:}\DecValTok{30}\NormalTok{, }\AttributeTok{nrow=}\DecValTok{5}\NormalTok{)}
\NormalTok{C }\OtherTok{\textless{}{-}} \FunctionTok{matrix}\NormalTok{(}\DecValTok{21}\SpecialCharTok{:}\DecValTok{40}\NormalTok{, }\AttributeTok{nrow=}\DecValTok{2}\NormalTok{)}

\FunctionTok{nrow}\NormalTok{(A)}
\end{Highlighting}
\end{Shaded}

\begin{verbatim}
## [1] 5
\end{verbatim}

\begin{Shaded}
\begin{Highlighting}[]
\FunctionTok{ncol}\NormalTok{(B)}
\end{Highlighting}
\end{Shaded}

\begin{verbatim}
## [1] 2
\end{verbatim}

\begin{Shaded}
\begin{Highlighting}[]
\FunctionTok{dim}\NormalTok{(C)}
\end{Highlighting}
\end{Shaded}

\begin{verbatim}
## [1]  2 10
\end{verbatim}

\begin{Shaded}
\begin{Highlighting}[]
\NormalTok{A }\SpecialCharTok{+}\NormalTok{ B}
\end{Highlighting}
\end{Shaded}

\begin{verbatim}
##      [,1] [,2]
## [1,]   22   32
## [2,]   24   34
## [3,]   26   36
## [4,]   28   38
## [5,]   30   40
\end{verbatim}

\begin{Shaded}
\begin{Highlighting}[]
\NormalTok{A }\SpecialCharTok{*}\NormalTok{ B}
\end{Highlighting}
\end{Shaded}

\begin{verbatim}
##      [,1] [,2]
## [1,]   21  156
## [2,]   44  189
## [3,]   69  224
## [4,]   96  261
## [5,]  125  300
\end{verbatim}

\begin{Shaded}
\begin{Highlighting}[]
\NormalTok{A }\SpecialCharTok{{-}}\NormalTok{ B}
\end{Highlighting}
\end{Shaded}

\begin{verbatim}
##      [,1] [,2]
## [1,]  -20  -20
## [2,]  -20  -20
## [3,]  -20  -20
## [4,]  -20  -20
## [5,]  -20  -20
\end{verbatim}

\begin{Shaded}
\begin{Highlighting}[]
\NormalTok{A }\OtherTok{=}\NormalTok{ B}
\end{Highlighting}
\end{Shaded}

\subsection{Matrix Multiplication and names in
R.}\label{matrix-multiplication-and-names-in-r.}

\begin{Shaded}
\begin{Highlighting}[]
\CommentTok{\# A \%*\% C will work}
\CommentTok{\# A \%*\% B will not work}
\CommentTok{\# Both A and B are 5 x 2 matrices}
\CommentTok{\# so we will transpose B}
\NormalTok{A }\SpecialCharTok{\%*\%} \FunctionTok{t}\NormalTok{(B)}
\end{Highlighting}
\end{Shaded}

\begin{verbatim}
##      [,1] [,2] [,3] [,4] [,5]
## [1,] 1117 1164 1211 1258 1305
## [2,] 1164 1213 1262 1311 1360
## [3,] 1211 1262 1313 1364 1415
## [4,] 1258 1311 1364 1417 1470
## [5,] 1305 1360 1415 1470 1525
\end{verbatim}

\begin{Shaded}
\begin{Highlighting}[]
\CommentTok{\#Column/row names of matrix:}
\FunctionTok{colnames}\NormalTok{(A)}
\end{Highlighting}
\end{Shaded}

\begin{verbatim}
## NULL
\end{verbatim}

\begin{Shaded}
\begin{Highlighting}[]
\FunctionTok{colnames}\NormalTok{(A) }\OtherTok{\textless{}{-}} \FunctionTok{c}\NormalTok{(}\StringTok{"Left"}\NormalTok{, }\StringTok{"Right"}\NormalTok{)}
\FunctionTok{rownames}\NormalTok{(A) }\OtherTok{\textless{}{-}} \FunctionTok{c}\NormalTok{(}\StringTok{"1st"}\NormalTok{, }\StringTok{"2nd"}\NormalTok{,}
\StringTok{"3rd"}\NormalTok{, }\StringTok{"4th"}\NormalTok{, }\StringTok{"5th"}\NormalTok{)}
\FunctionTok{t}\NormalTok{(A)}
\end{Highlighting}
\end{Shaded}

\begin{verbatim}
##       1st 2nd 3rd 4th 5th
## Left   21  22  23  24  25
## Right  26  27  28  29  30
\end{verbatim}

\begin{Shaded}
\begin{Highlighting}[]
\FunctionTok{colnames}\NormalTok{(B) }\OtherTok{\textless{}{-}} \FunctionTok{c}\NormalTok{(}\StringTok{"First"}\NormalTok{,}
\StringTok{"Second"}\NormalTok{)}
\FunctionTok{rownames}\NormalTok{(B) }\OtherTok{\textless{}{-}} \FunctionTok{c}\NormalTok{(}\StringTok{"One"}\NormalTok{, }\StringTok{"Two"}\NormalTok{,}
\StringTok{"Three"}\NormalTok{, }\StringTok{"Four"}\NormalTok{, }\StringTok{"Five"}\NormalTok{)}
\end{Highlighting}
\end{Shaded}

\subsection{Arrays in R.}\label{arrays-in-r.}

\begin{Shaded}
\begin{Highlighting}[]
\NormalTok{theArray }\OtherTok{\textless{}{-}} \FunctionTok{array}\NormalTok{(}\DecValTok{1}\SpecialCharTok{:}\DecValTok{12}\NormalTok{,}\AttributeTok{dim=}\FunctionTok{c}\NormalTok{(}\DecValTok{2}\NormalTok{,}\DecValTok{3}\NormalTok{,}\DecValTok{2}\NormalTok{))}
\CommentTok{\# 2 dimensional matrices both with 2 rows and 3 columns}

\NormalTok{theArray [}\DecValTok{1}\NormalTok{, , ] }\CommentTok{\# 1st row of both}
\end{Highlighting}
\end{Shaded}

\begin{verbatim}
##      [,1] [,2]
## [1,]    1    7
## [2,]    3    9
## [3,]    5   11
\end{verbatim}

\begin{Shaded}
\begin{Highlighting}[]
\NormalTok{theArray[}\DecValTok{1}\NormalTok{, ,}\DecValTok{1}\NormalTok{] }\CommentTok{\#1st row of first}
\end{Highlighting}
\end{Shaded}

\begin{verbatim}
## [1] 1 3 5
\end{verbatim}

\begin{Shaded}
\begin{Highlighting}[]
\NormalTok{theArray[,}\DecValTok{1}\NormalTok{,] }\CommentTok{\# 1st column of both}
\end{Highlighting}
\end{Shaded}

\begin{verbatim}
##      [,1] [,2]
## [1,]    1    7
## [2,]    2    8
\end{verbatim}

\subsection{Load the iris data from
Internet.}\label{load-the-iris-data-from-internet.}

\begin{Shaded}
\begin{Highlighting}[]
\NormalTok{iris }\OtherTok{\textless{}{-}} \FunctionTok{read.csv}\NormalTok{(}\FunctionTok{url}\NormalTok{(}\StringTok{"http://archive.ics.uci.edu/ml/machine{-}learning{-}databases/iris/iris.data"}\NormalTok{), }\AttributeTok{header =} \ConstantTok{FALSE}\NormalTok{)}
\FunctionTok{head}\NormalTok{(iris)}
\end{Highlighting}
\end{Shaded}

\begin{verbatim}
##    V1  V2  V3  V4          V5
## 1 5.1 3.5 1.4 0.2 Iris-setosa
## 2 4.9 3.0 1.4 0.2 Iris-setosa
## 3 4.7 3.2 1.3 0.2 Iris-setosa
## 4 4.6 3.1 1.5 0.2 Iris-setosa
## 5 5.0 3.6 1.4 0.2 Iris-setosa
## 6 5.4 3.9 1.7 0.4 Iris-setosa
\end{verbatim}

\begin{Shaded}
\begin{Highlighting}[]
\CommentTok{\# Add column names for V1, V2, V3, V4 and V5 columns to the Iris data}
\FunctionTok{names}\NormalTok{(iris) }\OtherTok{\textless{}{-}} \FunctionTok{c}\NormalTok{(}\StringTok{"Sepal.Length"}\NormalTok{, }\StringTok{"Sepal.Width"}\NormalTok{, }\StringTok{"Petal.Length"}\NormalTok{,}
\StringTok{"Petal.Width"}\NormalTok{, }\StringTok{"Species"}\NormalTok{)}
\FunctionTok{write.csv}\NormalTok{(iris, }\StringTok{"iris.csv"}\NormalTok{)}
\FunctionTok{library}\NormalTok{(magrittr) }\CommentTok{\#for pipes}
\FunctionTok{sink}\NormalTok{(}\StringTok{"session3.out"}\NormalTok{)}
\FunctionTok{plot}\NormalTok{(iris)}
\end{Highlighting}
\end{Shaded}

\includegraphics{Unit-1-Project_files/figure-latex/unnamed-chunk-35-1.pdf}

\begin{Shaded}
\begin{Highlighting}[]
\FunctionTok{summary}\NormalTok{(iris)}
\NormalTok{iris}\SpecialCharTok{$}\NormalTok{Sepal.Length.SQRT }\OtherTok{\textless{}{-}}\NormalTok{ iris}\SpecialCharTok{$}\NormalTok{Sepal.Length }\SpecialCharTok{\%\textgreater{}\%} \FunctionTok{sqrt}\NormalTok{()}
\CommentTok{\#iris \%\textgreater{}\% cor(Sepal.Length, Sepal.Width)}
\FunctionTok{sink}\NormalTok{()}
\FunctionTok{detach}\NormalTok{(}\StringTok{"package:magrittr"}\NormalTok{)}
\end{Highlighting}
\end{Shaded}

\section{Session-5 code Examples and
Exercises}\label{session-5-code-examples-and-exercises}

\begin{Shaded}
\begin{Highlighting}[]
\CommentTok{\# Built In Functions in R.}
\CommentTok{\#round() }
\FunctionTok{print}\NormalTok{(}\FunctionTok{round}\NormalTok{(}\FloatTok{3.1415}\NormalTok{)) }
\end{Highlighting}
\end{Shaded}

\begin{verbatim}
## [1] 3
\end{verbatim}

\begin{Shaded}
\begin{Highlighting}[]
\FunctionTok{round}\NormalTok{(}\FloatTok{3.1415}\NormalTok{, }\AttributeTok{digits =} \DecValTok{2}\NormalTok{)}
\end{Highlighting}
\end{Shaded}

\begin{verbatim}
## [1] 3.14
\end{verbatim}

\begin{Shaded}
\begin{Highlighting}[]
\CommentTok{\#factorial()}
\FunctionTok{factorial}\NormalTok{(}\DecValTok{3}\NormalTok{) }
\end{Highlighting}
\end{Shaded}

\begin{verbatim}
## [1] 6
\end{verbatim}

\begin{Shaded}
\begin{Highlighting}[]
\FunctionTok{factorial}\NormalTok{(}\DecValTok{2}\SpecialCharTok{*}\DecValTok{3}\NormalTok{)}
\end{Highlighting}
\end{Shaded}

\begin{verbatim}
## [1] 720
\end{verbatim}

\begin{Shaded}
\begin{Highlighting}[]
\CommentTok{\#mean()}
\FunctionTok{mean}\NormalTok{(}\DecValTok{1}\SpecialCharTok{:}\DecValTok{6}\NormalTok{) }
\end{Highlighting}
\end{Shaded}

\begin{verbatim}
## [1] 3.5
\end{verbatim}

\begin{Shaded}
\begin{Highlighting}[]
\FunctionTok{mean}\NormalTok{(}\FunctionTok{c}\NormalTok{(}\DecValTok{1}\SpecialCharTok{:}\DecValTok{30}\NormalTok{))}
\end{Highlighting}
\end{Shaded}

\begin{verbatim}
## [1] 15.5
\end{verbatim}

\subsection{``Sample'' function: Random sampling without or with
replacement in
R}\label{sample-function-random-sampling-without-or-with-replacement-in-r}

\begin{Shaded}
\begin{Highlighting}[]
\NormalTok{die }\OtherTok{\textless{}{-}} \DecValTok{1}\SpecialCharTok{:}\DecValTok{6}
\FunctionTok{sample}\NormalTok{(}\AttributeTok{x =}\NormalTok{ die, }\AttributeTok{size =} \DecValTok{1}\NormalTok{)}
\end{Highlighting}
\end{Shaded}

\begin{verbatim}
## [1] 6
\end{verbatim}

\begin{Shaded}
\begin{Highlighting}[]
\FunctionTok{sample}\NormalTok{(}\AttributeTok{x =}\NormalTok{ die, }\AttributeTok{size =} \DecValTok{1}\NormalTok{)}
\end{Highlighting}
\end{Shaded}

\begin{verbatim}
## [1] 6
\end{verbatim}

\begin{Shaded}
\begin{Highlighting}[]
\FunctionTok{sample}\NormalTok{ (}\AttributeTok{x =}\NormalTok{ die, }\AttributeTok{size =} \DecValTok{1}\NormalTok{, }\AttributeTok{replace=}\ConstantTok{TRUE}\NormalTok{)}
\end{Highlighting}
\end{Shaded}

\begin{verbatim}
## [1] 1
\end{verbatim}

\begin{Shaded}
\begin{Highlighting}[]
\FunctionTok{sample}\NormalTok{(}\AttributeTok{x =}\NormalTok{ die, }\AttributeTok{size =} \DecValTok{2}\NormalTok{)}
\end{Highlighting}
\end{Shaded}

\begin{verbatim}
## [1] 6 1
\end{verbatim}

\begin{Shaded}
\begin{Highlighting}[]
\FunctionTok{sample}\NormalTok{(x }\SpecialCharTok{{-}}\NormalTok{ die, }\AttributeTok{size =} \DecValTok{2}\NormalTok{)}
\end{Highlighting}
\end{Shaded}

\begin{verbatim}
## Warning in x - die: longer object length is not a multiple of shorter object
## length
\end{verbatim}

\begin{verbatim}
## [1] 3 9
\end{verbatim}

\begin{Shaded}
\begin{Highlighting}[]
\FunctionTok{sample}\NormalTok{(}\AttributeTok{x =}\NormalTok{ die, }\AttributeTok{size =} \DecValTok{2}\NormalTok{, }\AttributeTok{replace=}\ConstantTok{TRUE}\NormalTok{)}
\end{Highlighting}
\end{Shaded}

\begin{verbatim}
## [1] 6 3
\end{verbatim}

\subsection{User Defined Functions in
R}\label{user-defined-functions-in-r}

\begin{Shaded}
\begin{Highlighting}[]
\NormalTok{roll }\OtherTok{\textless{}{-}} \ControlFlowTok{function}\NormalTok{() \{}
\NormalTok{die }\OtherTok{\textless{}{-}} \DecValTok{1}\SpecialCharTok{:}\DecValTok{6}
\NormalTok{dice }\OtherTok{\textless{}{-}} \FunctionTok{sample}\NormalTok{(die, }\AttributeTok{size =} \DecValTok{2}\NormalTok{, }\AttributeTok{replace =} \ConstantTok{TRUE}\NormalTok{)}
\FunctionTok{sum}\NormalTok{(dice)}
\NormalTok{\}}
\CommentTok{\# Function with arguments with default values}
\NormalTok{roll2 }\OtherTok{\textless{}{-}} \ControlFlowTok{function}\NormalTok{(}\AttributeTok{dice =} \DecValTok{1}\SpecialCharTok{:}\DecValTok{6}\NormalTok{) \{}

\NormalTok{dice }\OtherTok{\textless{}{-}} \FunctionTok{sample}\NormalTok{(dice, }\AttributeTok{size =} \DecValTok{2}\NormalTok{, }\AttributeTok{replace =} \ConstantTok{TRUE}\NormalTok{)}
\FunctionTok{sum}\NormalTok{(dice)}

\NormalTok{\}}

\FunctionTok{roll}\NormalTok{()}
\end{Highlighting}
\end{Shaded}

\begin{verbatim}
## [1] 7
\end{verbatim}

\begin{Shaded}
\begin{Highlighting}[]
\FunctionTok{roll}\NormalTok{()}
\end{Highlighting}
\end{Shaded}

\begin{verbatim}
## [1] 7
\end{verbatim}

\begin{Shaded}
\begin{Highlighting}[]
\FunctionTok{roll}\NormalTok{()}
\end{Highlighting}
\end{Shaded}

\begin{verbatim}
## [1] 4
\end{verbatim}

\begin{Shaded}
\begin{Highlighting}[]
\FunctionTok{roll2}\NormalTok{()}
\end{Highlighting}
\end{Shaded}

\begin{verbatim}
## [1] 11
\end{verbatim}

\begin{Shaded}
\begin{Highlighting}[]
\FunctionTok{roll2}\NormalTok{()}
\end{Highlighting}
\end{Shaded}

\begin{verbatim}
## [1] 7
\end{verbatim}

\begin{Shaded}
\begin{Highlighting}[]
\FunctionTok{roll2}\NormalTok{()}
\end{Highlighting}
\end{Shaded}

\begin{verbatim}
## [1] 7
\end{verbatim}

\subsection{Loops in R.}\label{loops-in-r.}

\begin{Shaded}
\begin{Highlighting}[]
\CommentTok{\# for loop}
\CommentTok{\#while loop}
\NormalTok{print\_words }\OtherTok{\textless{}{-}} \ControlFlowTok{function}\NormalTok{(sentence) \{}
\ControlFlowTok{for}\NormalTok{ (word }\ControlFlowTok{in}\NormalTok{ sentence) \{}
\FunctionTok{print}\NormalTok{(word)}
\NormalTok{\}}
\NormalTok{\}}

\NormalTok{best\_practice }\OtherTok{\textless{}{-}} \FunctionTok{c}\NormalTok{(}\StringTok{"Let"}\NormalTok{, }\StringTok{"the"}\NormalTok{, }\StringTok{"computer"}\NormalTok{, }\StringTok{"do"}\NormalTok{, }\StringTok{"the"}\NormalTok{, }\StringTok{"work"}\NormalTok{)}
\FunctionTok{print\_words}\NormalTok{(best\_practice)}
\end{Highlighting}
\end{Shaded}

\begin{verbatim}
## [1] "Let"
## [1] "the"
## [1] "computer"
## [1] "do"
## [1] "the"
## [1] "work"
\end{verbatim}

\begin{Shaded}
\begin{Highlighting}[]
\FunctionTok{print\_words}\NormalTok{(best\_practice[}\SpecialCharTok{{-}}\DecValTok{6}\NormalTok{])}
\end{Highlighting}
\end{Shaded}

\begin{verbatim}
## [1] "Let"
## [1] "the"
## [1] "computer"
## [1] "do"
## [1] "the"
\end{verbatim}

\subsection{Condition If and Else}\label{condition-if-and-else}

\#Checking values of y with x:

\begin{Shaded}
\begin{Highlighting}[]
\ControlFlowTok{if}\NormalTok{ (y }\SpecialCharTok{\textless{}} \DecValTok{20}\NormalTok{) \{}
\NormalTok{x }\OtherTok{\textless{}{-}} \StringTok{"Too low"}
\NormalTok{\} }\ControlFlowTok{else}\NormalTok{ \{}
\NormalTok{x }\OtherTok{\textless{}{-}} \StringTok{"Too high"}
\NormalTok{\}}
\end{Highlighting}
\end{Shaded}

\begin{verbatim}
## Warning in if (y < 20) {: the condition has length > 1 and only the first
## element will be used
\end{verbatim}

\begin{Shaded}
\begin{Highlighting}[]
\CommentTok{\#Can you get anything from this?}

\CommentTok{\#Will this work?}
\NormalTok{check.y }\OtherTok{\textless{}{-}} \ControlFlowTok{function}\NormalTok{(y) \{}
\ControlFlowTok{if}\NormalTok{ (y }\SpecialCharTok{\textless{}} \DecValTok{20}\NormalTok{) \{}
\FunctionTok{print}\NormalTok{(}\StringTok{"Too Low"}\NormalTok{) \} }\ControlFlowTok{else}\NormalTok{ \{}
\FunctionTok{print}\NormalTok{(}\StringTok{"Two high"}\NormalTok{)}
\NormalTok{\}\}}

\FunctionTok{check.y}\NormalTok{(}\DecValTok{10}\NormalTok{)}
\end{Highlighting}
\end{Shaded}

\begin{verbatim}
## [1] "Too Low"
\end{verbatim}

\begin{Shaded}
\begin{Highlighting}[]
\FunctionTok{check.y}\NormalTok{(}\DecValTok{30}\NormalTok{)}
\end{Highlighting}
\end{Shaded}

\begin{verbatim}
## [1] "Two high"
\end{verbatim}

\subsection{Creating binary variables with
``ifelse''}\label{creating-binary-variables-with-ifelse}

\begin{Shaded}
\begin{Highlighting}[]
\CommentTok{\#Will this work?}
\NormalTok{y }\OtherTok{\textless{}{-}} \DecValTok{1}\SpecialCharTok{:}\DecValTok{40}
\FunctionTok{ifelse}\NormalTok{(y}\SpecialCharTok{\textless{}}\DecValTok{20}\NormalTok{, }\StringTok{"Too low"}\NormalTok{, }\StringTok{"Too high"}\NormalTok{)}
\end{Highlighting}
\end{Shaded}

\begin{verbatim}
##  [1] "Too low"  "Too low"  "Too low"  "Too low"  "Too low"  "Too low" 
##  [7] "Too low"  "Too low"  "Too low"  "Too low"  "Too low"  "Too low" 
## [13] "Too low"  "Too low"  "Too low"  "Too low"  "Too low"  "Too low" 
## [19] "Too low"  "Too high" "Too high" "Too high" "Too high" "Too high"
## [25] "Too high" "Too high" "Too high" "Too high" "Too high" "Too high"
## [31] "Too high" "Too high" "Too high" "Too high" "Too high" "Too high"
## [37] "Too high" "Too high" "Too high" "Too high"
\end{verbatim}

\begin{Shaded}
\begin{Highlighting}[]
\CommentTok{\#It’s a logical as:}
\FunctionTok{ifelse}\NormalTok{(y}\SpecialCharTok{\textless{}}\DecValTok{20}\NormalTok{, }\ConstantTok{TRUE}\NormalTok{, }\ConstantTok{FALSE}\NormalTok{)}
\end{Highlighting}
\end{Shaded}

\begin{verbatim}
##  [1]  TRUE  TRUE  TRUE  TRUE  TRUE  TRUE  TRUE  TRUE  TRUE  TRUE  TRUE  TRUE
## [13]  TRUE  TRUE  TRUE  TRUE  TRUE  TRUE  TRUE FALSE FALSE FALSE FALSE FALSE
## [25] FALSE FALSE FALSE FALSE FALSE FALSE FALSE FALSE FALSE FALSE FALSE FALSE
## [37] FALSE FALSE FALSE FALSE
\end{verbatim}

\begin{Shaded}
\begin{Highlighting}[]
\NormalTok{y }\OtherTok{\textless{}{-}} \DecValTok{1}\SpecialCharTok{:}\DecValTok{40}
\FunctionTok{ifelse}\NormalTok{(y}\SpecialCharTok{\textless{}}\DecValTok{20}\NormalTok{, }\DecValTok{1}\NormalTok{, }\DecValTok{0}\NormalTok{)}
\end{Highlighting}
\end{Shaded}

\begin{verbatim}
##  [1] 1 1 1 1 1 1 1 1 1 1 1 1 1 1 1 1 1 1 1 0 0 0 0 0 0 0 0 0 0 0 0 0 0 0 0 0 0 0
## [39] 0 0
\end{verbatim}

\subsection{Multiple Conditions: combining
``ifelse''}\label{multiple-conditions-combining-ifelse}

\begin{Shaded}
\begin{Highlighting}[]
\CommentTok{\# Will this work too?}
\NormalTok{x }\OtherTok{\textless{}{-}} \DecValTok{1}\SpecialCharTok{:}\DecValTok{99}

\CommentTok{\# Binary numbers}
\NormalTok{x1 }\OtherTok{\textless{}{-}} \FunctionTok{ifelse}\NormalTok{(x }\SpecialCharTok{\textless{}} \DecValTok{20}\NormalTok{, }\DecValTok{1}\NormalTok{, }\DecValTok{0}\NormalTok{) }

\CommentTok{\# Binary text}
\NormalTok{x2}\FloatTok{.1} \OtherTok{\textless{}{-}} \FunctionTok{ifelse}\NormalTok{(x }\SpecialCharTok{\textless{}} \DecValTok{20}\NormalTok{, }\StringTok{"\textless{}20"}\NormalTok{, }\StringTok{"20+"}\NormalTok{)  }

\CommentTok{\# Define categorical variables for different ranges}
\NormalTok{x2}\FloatTok{.2} \OtherTok{\textless{}{-}} \FunctionTok{ifelse}\NormalTok{(x }\SpecialCharTok{\textgreater{}=} \DecValTok{20} \SpecialCharTok{\&}\NormalTok{ x }\SpecialCharTok{\textless{}} \DecValTok{40}\NormalTok{, }\StringTok{"20{-}39"}\NormalTok{, }\ConstantTok{NA}\NormalTok{)  }
\NormalTok{x2}\FloatTok{.3} \OtherTok{\textless{}{-}} \FunctionTok{ifelse}\NormalTok{(x }\SpecialCharTok{\textgreater{}=} \DecValTok{40} \SpecialCharTok{\&}\NormalTok{ x }\SpecialCharTok{\textless{}} \DecValTok{100}\NormalTok{, }\StringTok{"40{-}99"}\NormalTok{, }\ConstantTok{NA}\NormalTok{)  }

\CommentTok{\# Now combine them in a single column with \textless{}20 =1, 20{-}39 = 2 and 40{-}99 = 3}
\CommentTok{\# Create categorical variable of x}
\NormalTok{x3 }\OtherTok{\textless{}{-}} \FunctionTok{ifelse}\NormalTok{(x }\SpecialCharTok{\textless{}} \DecValTok{20}\NormalTok{, }\DecValTok{1}\NormalTok{, }\FunctionTok{ifelse}\NormalTok{(x }\SpecialCharTok{\textless{}} \DecValTok{40}\NormalTok{, }\DecValTok{2}\NormalTok{, }\DecValTok{3}\NormalTok{))}

\CommentTok{\# Output the categorical variable}
\NormalTok{x3}
\end{Highlighting}
\end{Shaded}

\begin{verbatim}
##  [1] 1 1 1 1 1 1 1 1 1 1 1 1 1 1 1 1 1 1 1 2 2 2 2 2 2 2 2 2 2 2 2 2 2 2 2 2 2 2
## [39] 2 3 3 3 3 3 3 3 3 3 3 3 3 3 3 3 3 3 3 3 3 3 3 3 3 3 3 3 3 3 3 3 3 3 3 3 3 3
## [77] 3 3 3 3 3 3 3 3 3 3 3 3 3 3 3 3 3 3 3 3 3 3 3
\end{verbatim}

\begin{Shaded}
\begin{Highlighting}[]
\CommentTok{\# Display frequency count of categories}
\FunctionTok{table}\NormalTok{(x3)}
\end{Highlighting}
\end{Shaded}

\begin{verbatim}
## x3
##  1  2  3 
## 19 20 60
\end{verbatim}

\end{document}
